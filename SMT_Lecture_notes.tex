\documentclass[a4paper, 14pt]{extarticle}

\usepackage[english, russian]{babel}
\usepackage{fancyhdr, fancybox}
\usepackage{geometry}
\usepackage{amssymb}
\usepackage{amsthm}
\usepackage{amsmath}\usepackage{mathtools}
\usepackage{amssymb}
\usepackage{amsthm}
\usepackage{hyperref}
\usepackage{tikz}

\geometry{a4paper,total={170mm,257mm},left=1cm,right=1cm,top=2cm,bottom=1.5cm}

%opening
\title{Факультатив <<Структурная теория матриц>>}

% Клонтитулы
\fancyhead[R]{\textbf{Структурная теория матриц}}
\fancyhead[L]{\textbf{НИУ ВШЭ}}

% Теоремы
\theoremstyle{definition}
\newtheorem{definition}{Определение}
\newtheorem{theorem}{Теорема}
\newtheorem{note}{Замечание}
\newtheorem{lemma}{Лемма}
\newtheorem{conseq}{Следствие}
\newtheorem{example}{Пример}
\newtheorem{state}{Утверждение}
\newtheorem{exec}{Упражнение}

\setcounter{tocdepth}{5}

\begin{document}
\pagestyle{fancy}
\maketitle

\newpage
\tableofcontents

\newpage

\section{Первая лекция}

\textbf{Вопрос курса:} Каким образом можно описать структуру конечномерных алгебр, используя матрицы, и какими свойствами будут обладать данные описания?
\begin{definition}
	 Пусть \(\mathcal{A}\) - векторное пространство над полем \(\mathbb{F}\) снабжённое операцией \[\cdot : \mathcal{A} \times \mathcal{A} \rightarrow \mathcal{A} \text{ (умножение)}\]
	 тогда \(\mathcal{A}\) является алгеброй, если \(\forall x, y, z \in \mathcal{A}\) и \(\forall a, b \in \mathbb{F}\) верно:
	 \begin{itemize}
	 	\item \(x(y + z) = xy + xz\)
	 	\item \((x + y)z = xz + yz\)
	 	\item \((ax) \cdot (by) = (ab)(x \cdot y)\)
	 \end{itemize}
\end{definition}

\begin{note}
	В этом курсе рассматриваются ассоциативные алгебры, если не оговорено иное.
\end{note}

\begin{example}
	\leavevmode
	\begin{enumerate}
		\item Многочлены \(\mathbb{F}[x]\); \(\operatorname{dim}\mathbb{F}[x] = \infty\)
		\item Алгебра кватернионов: \[\mathbb{H} = \{a + bi + cj + dk | a, b, c, d \in \mathbb{R}; i^2 = j^2 = k^2 = ijk = -1\}\]
		\item Групповые алгебры. Рассмотрим группу \(G\) и поле \(\mathbb{F}\), тогда: \[\mathbb{F}[G]= \left\{\sum\limits_{i = 1}^{|G|} \lambda_i g_i \left|\right. \lambda_i \in \mathbb{F}, g_i \in G \right\}; \operatorname{dim} \mathbb{F}[G] = |G|\]
		\item Матричные алгебры: Пусть \(\mathcal{A} \subseteq M_n(\mathbb{F})\) и \(\forall a, b \in \mathcal{A};\ \forall \lambda \in \mathbb{F}\) замкнуто по операциям:
		\begin{itemize}
			\item \(a + b \in \mathcal{A}\)
			\item \(\lambda a \in \mathcal{A}\)
			\item \(a \cdot b \in \mathcal{A}\)
		\end{itemize}
	\end{enumerate}
\end{example}

\paragraph{Матричное представление конечномерных алгебр}

\begin{state}
	Любая конечномерная алгебра изоморфна некоторой конечной матричной алгебре.
\end{state}

Пусть \(\mathcal{A}\) - алгебра, \(\{e_1, \dots, e_n\}\) - базис в \(\mathcal{A}\)
Умножение элементов \(\mathcal{A}\) слева на \(e_i\) является линейным отображением \(\forall a, b \in \mathcal{A};\ \forall \lambda \in \mathbb{F}\):
\begin{itemize}
	\item \(e_i(a + b) = e_i a + e_i b\)
	\item \(e_i(\lambda a) = \lambda(e_i a)\)
\end{itemize}
\(e_i\) можно поставить в соответствие матрицу \(A_{e_i}\) в базисе \(\{e_1, \dots, e_n\}\). При этом:

\begin{itemize}
	\item \(A_{a + b} = A_a + A_b\)
	\item \(A_{ab} = A_a \cdot A_b\)
	\item \(A_{\lambda a} = \lambda A_a\)
\end{itemize}

\begin{example}
	\(\mathbb{H}\) с базисом \(\{1, i, j, k\}\)
	\[A_1 = E;\ A_i = \left( 
	\begin{array}{cccc}
	0&-1&0&0\\
	1&0&0&0\\
	0&0&0&-1\\
	0&0&1&0
	\end{array} 
	\right);\ A_j = \left( 
	\begin{array}{cccc}
		0&0&-1&0\\
		0&0&0&1\\
		1&0&0&0\\
		0&-1&0&0
	\end{array} 
	\right);\ A_k = \left( 
	\begin{array}{cccc}
		0&0&0&-1\\
		0&0&-1&0\\
		0&1&0&0\\
		1&0&0&0
	\end{array} 
	\right)\]
	\[\mathbb{H} = \{a + bi + cj + dk | a, b, c, d \in \mathbb{R}\} \simeq \left\{
		\left(
			\left. \begin{array}{cccc}
				a&-b&-c&-d\\
				b&a&-d&c\\
				c&d&a&-b\\
				d&-c&b&a
			\end{array}
		\right) \right| a, b, c, d \in \mathbb{R}
	\right\}\]
\end{example}

\paragraph{Матричные алгебры}

\begin{example}
	\leavevmode
	\begin{enumerate}
		\item (Блочно-)диагональные матрицы:
		\[\left(
			\begin{array}{ccc}
				a_{11}&\hdots&0\\
				\vdots&\ddots&\vdots\\
				0&\dots&a_{nn}
			\end{array}
		\right) a_{ii} \in \mathbb{F},\ \operatorname{dim} \mathcal{A} = n\]
		\[\left(
			\begin{array}{c|c|c}
				A_{11}&\hdots&0\\
				\hline
				\vdots&\ddots&\vdots\\
				\hline
				0&\dots&A_{nn}
			\end{array}
		\right) A_{ii} \in M_{n_i}(\mathbb{F}),\ \operatorname{dim} \mathcal{A} = \sum\limits_{i=1}^n n^2_i\]
		
		\item (Блочно-)верхнетреугольные матрицы:
		\[
			\left(
				\begin{array}{ccc}
					a_{11}&&*\\
					&\ddots&\\
					0&&a_{nn}
				\end{array}
			\right) \text{ если \(j > i\) то } a_{ij} = 0 \text{, иначе } a_{ij} \in \mathbb{F},\ \operatorname{dim} \mathcal{A} = \frac{n(n + 1)}{2}
		\]
		\[
			\left(
				\begin{array}{c|c|c|c}
					A_{11}&*&\hdots&*\\
					\hline
					0&A_{22}&\hdots&*\\
					\hline
					\vdots&\vdots&\ddots&\vdots\\
					\hline
					0&0&\hdots&A_{nn}
				\end{array}
			\right) A_{ii} \in M_n(\mathbb{F}),\ \operatorname{dim} \mathcal{A} = \frac{n^2 + \sum\limits_{i=1}^n n_i^2}{2}
		\]
		
		\item Алгебра порожденная некоторым \(S = \{a_1, \dots, a_k\} \subseteq M_n(\mathbb{F}).\)
		\[\left. \mathcal{L}(S) = \left\{ \sum\limits_{i=1}^{t \leqslant k} \lambda_i a_i \right| \lambda_i \in \mathbb{F},\ a_i \in S \right\}\]
	\end{enumerate}
\end{example}

\begin{exec}
	Является ли алгеброй: \[\mathcal{A} = \left\{ A \in M_n(\mathbb{F}) \left| \sum \right. a_{ij} = 0,\ n \geqslant 2 \right\} ?\]
\end{exec}

\begin{proof}[Решение]
	Нет, не является. Рассмотрим оба возможных случая:
	\begin{itemize}
		\item \(\operatorname{char} \mathbb{F} \neq 2:\)
		\[\mathcal{A} \ni a = \left(
			\begin{array}{cc}
				1&0\\
				0&-1
			\end{array}
		\right) \text{, но } a^2 = \left(
		\begin{array}{cc}
			1&0\\
			0&1
		\end{array}
		\right) \notin \mathcal{A}\]
		
		\item \(\operatorname{char} \mathbb{F} = 2:\)
		\[b = \left(
			\begin{array}{ccc}
				0&0&1\\
				0&0&1\\
				1&0&1\\
			\end{array}
		\right),\ c = \left(
		\begin{array}{ccc}
			1&0&0\\
			1&0&0\\
			0&0&0\\
		\end{array}
		\right),\ bc = \left(
		\begin{array}{ccc}
			0&0&0\\
			0&0&0\\
			1&0&0\\
		\end{array}
		\right) \text{, где } b, c \in \mathcal{A},\ bc \notin \mathcal{A}\]
	\end{itemize}
\end{proof}

\begin{exec}\label{ex1_2}
	Является ли алгеброй множество всех матриц, с одним фиксированным собственным вектором \(v \neq 0\)?
\end{exec}

\begin{proof}[Решение]
	Да, является. Рассмотрим \(A, B\) такие что: \(Av = \lambda v\) и \(Bv = \mu v:\)
	\[(A + B)v = (\lambda + \mu)v;\ ABv = \lambda\mu v;\ (\tau A)v = \tau\lambda v\]
\end{proof}

\begin{exec}
	Найти размерность алгебры из \hyperref[ex1_2]{\textbf{Упр. 2}}.
\end{exec}

\begin{proof}[Решение]
	Переведем матрицы в алгебре из \hyperref[ex1_2]{\textbf{Упр. 2}} в базис \(\{v, e_1, \dots, e_n\}\), тогда:
	\[\mathcal{A}_v = \left\{
		\left(
			\left. 
				\begin{array}{cccc}
					\lambda&a_{12}&\hdots&a_{1n}\\
					0&a_{22}&\hdots&a_{2n}\\
					\vdots&\vdots&\ddots&\vdots\\
					0&a_{2n}&\hdots&a_{nn}
				\end{array}
			 \right) \right| \lambda \in \mathbb{F}
	\right\} \Rightarrow \operatorname{dim} \mathcal{A}_v = n^2 - n + 1\]
\end{proof}

\begin{exec}
	Найти \(k\)-мерную подалгебру в алгебре \(\operatorname{M}_4(\mathbb{R})\).
	\begin{itemize}
		\item \(k = 7\)
		\item \(k = 12\) и в матрицах нет полностью нулевых строк/столбцов.
	\end{itemize}
\end{exec}

\begin{proof}[Решение]
	\leavevmode
	\begin{itemize}
		\item \(k = 7\)
		\[\left\{
		\left(
		\left. \begin{array}{cccc}
			a&b&c&d\\
			0&0&e&f\\
			0&0&0&g\\
			0&0&0&0\\
		\end{array}
		\right) \right| a, b, c, d, e, f, g \in \mathbb{R}
		\right\}\]
		
		\item \(k = 12\) и в мтрицах нет полностью нулевых строк/столбцов.
			\[\left\{
		\left(
		\left. \begin{array}{cccc}
			a_{11}&a_{12}&a_{13}&a_{14}\\
			a_{21}&a_{22}&a_{23}&a_{24}\\
			a_{31}&a_{32}&a_{33}&a_{34}\\
			a_{31}&a_{32}&a_{33}&a_{34}\\
		\end{array}
		\right) \right| a_{ij} \in \mathbb{R}
		\right\}\]
	\end{itemize}
\end{proof}

\newpage

\section{Вторая лекция}

\begin{exec}
	Доказать, что множество циркулянтов:
	\[C_n = \left\{
		\left(
			\begin{array}{cccc}
				a_1&a_2&\hdots&a_n\\
				a_n&a_1&\hdots&a_{n-1}\\
				\vdots&\vdots&\ddots&\vdots\\
				a_{2}&a_3&\hdots&a_1
			\end{array}
		\right)
	\right\}\]
	является алгеброй. Найти \(\operatorname{dim} C_n(\mathbb{F})\)
\end{exec}

\begin{proof}[Решение]
	Найдем \(A \in \text{M}_n(\mathbb{F})\) такую что:
	\(C_n(\mathbb{F}) = \mathcal{L}(\{A\})\).
	\[A = \left(
		\begin{array}{cccc}
			0&1&\hdots&0\\
			0&\ddots&\ddots&\vdots\\
			\vdots&&\ddots&1\\
			0&0&\hdots&0
		\end{array}
	\right);\ A^2 = \left(
		\begin{array}{ccccc}
			0&0&1&\hdots&0\\
			0&\ddots&\ddots&\ddots&\vdots\\
			\vdots&&\ddots&\ddots&1\\
			\vdots&&&\ddots&0\\
			0&0&0&\hdots&0
		\end{array}
	\right);\ \dots;\ A^n = E \]
	\[\forall k \in \mathbb{N} : A^k \in \{A, \dots, A^n\} \text{, так как } A^n = E \Rightarrow \mathcal{L}(\{A\}) = \{A_1, \dots, A_n\} = C_n(\mathbb{F})\]
	\[A, \dots, A^n \text{ - линейно независимы.} \Rightarrow \operatorname{dim} C_n(\mathbb{F}) = n\]
\end{proof}

\paragraph{Инвариантные подпространства}

\begin{definition}
	Подпространство \(V \subseteq \mathbb{F}^n\) называется инвариантным относительно множества матриц \(\mathcal{A} \subseteq \operatorname{M}_n(\mathbb{F})\) (\(\mathcal{A}\) -- инвариантным) если \(\forall v \in V\) и \(\forall A \in \mathcal{A}: Av \in V\).
\end{definition}

\begin{example}
	\leavevmode
	\begin{itemize}
		\item Подпространства \(\{0\}\) и \(\mathbb{F}^n\ \mathcal{A}\) -- инвариантны для любого \(\mathcal{A}\).
		\item \(v \in \mathbb{F}^n: \langle v \rangle\ \mathcal{A}\) -- инвариантно \(\Leftrightarrow v\) - общий собственный вектор для матриц из \(\mathcal{A}\)
	\end{itemize}
\end{example}

\begin{exec}
	Пусть \(V\) - подпространство \(\mathbb{F}^n,\ \operatorname{dim} V = k:\)
	\[\mathcal{A}_v = \{A \in \operatorname{M}_n(\mathbb{F}) | V \mathcal{A} \text{ -- инваринтно}\}\]
	Доказать, что \(\mathcal{A}_v\) - алгебра. \(\operatorname{dim} \mathcal{A}_v\)?
\end{exec}

\begin{proof}[Решение]
	\(\{e_1, \dots, e_k\}\) - базис в \(V\), дополним до базиса в \(\mathbb{F}^n : \{e_1, \dots, e_k, \dots, e_n\}\). Тогда:
	\[\mathcal{A}_v = \left\{
		\left(
			\begin{array}{c|c}
				*&*\\
				\hline
				0&*
			\end{array}
		\right) \in \operatorname{M}_n(\mathbb{F})
	\right\} \Rightarrow \operatorname{dim} \mathcal{A}_v = n^2 - k(n-k)\]
	где \(\mathcal{A}_v \text{ - блочно-верхнетреугольная алгебра}.\)
\end{proof}

\begin{exec}
	Найти нетривиальное общее инвариантное подпространство для матриц \(C_n(\mathbb{F})\).
\end{exec}

\begin{proof}[Решение]
	\[v = \left(\begin{array}{c}
		1\\
		1\\
		\vdots\\
		1
	\end{array}\right) \text{ --- собственный вектор } \forall A \in C_n(\mathbb{F}) \Rightarrow\]
	\(\Rightarrow \langle v \rangle \text{ --- нетривиальное \(C_n(\mathbb{F})\) - инвариантное подпространство}\)
\end{proof}

\paragraph{Неприводимые множества матриц}

\begin{note}
	В этой лекции далее будем считать, что алгебра содержит \(1(E)\).
\end{note}

\begin{definition}
	Подмножество \(\mathcal{A} \subseteq \operatorname{M}_n(\mathbb{F})\) называется неприводимым, если не существует нетривиальных \(\mathcal{A}\) - инвариантных подпространств в \(\mathbb{F}^n\).
\end{definition}

\begin{lemma} \label{lm2_1}
	Пусть \(\mathcal{A}\) --- неприводимая подалгебра \(\operatorname{M}_n(\mathbb{F})\). Тогда \(\forall v \neq 0 \in \mathbb{F}^n :\) \[\mathcal{A}v = \{Av | A \in \mathcal{A}\} = \mathbb{F}^n.\]
\end{lemma}

\begin{proof}[Доказательство]
	\(\mathcal{A}v\) --- подпространство в \(\mathbb{F}^n\). Рассмотрим \(u \in \mathcal{A}v\). Для \(u\) верно: \[\exists B \in \mathcal{A}: u = Bv \Rightarrow \forall A \in \mathcal{A} : Au = \underbrace{AB}_{AB \in \mathcal{A}} \cdot\ v \in \mathcal{A}v\]
	тогда \(\mathcal{A}v\) - инвариантно, при этом:
	\[\mathcal{A}v \neq \{0\} \text{, так как \(E \in \mathcal{A} \Rightarrow\) так как \(\mathcal{A}\) - неприводима, то } \mathcal{A}v = \mathbb{F}^n.\]
\end{proof}

\begin{lemma} \label{lm2_2}
	Пусть \(\mathcal{A}\) --- неприводимая подалгебра \(\operatorname{M}_n(\mathbb{F})\), тогда \(\forall v \neq 0;\ v \in \mathbb{F}^n : \)
	\[v^T\mathcal{A} = \{v^TA | A \in \mathcal{A}\} = (\mathbb{F}^n)^T.\]
\end{lemma}

\begin{proof}[Доказательство]
	\[V = v^T\mathcal{A} \subset (\mathbb{F}^n)^T\]
	\[\operatorname{Ann}V = \{u \in V | Vu = 0\}\]
	так как \(V \neq (\mathbb{F}^n)^T\), то \(\operatorname{Ann}V\) --- нетривиальное подпространство \(\mathbb{F}^n\)
	\[\forall u \in \operatorname{Ann}V\ \forall A \in \mathcal{A}: V \cdot Au = v^T \underbrace{\mathcal{A} \cdot A}_{\mathcal{A}A = \mathcal{A}}u = 0 \Rightarrow Au \in \operatorname{Ann}V\]
	получаем, что: \(\operatorname{Ann}V\) нетривиальное \(\mathcal{A}\) - инвариантное подпространство \(\mathbb{F}^n\), но \(\mathcal{A}\) --- неприводимая алгебра \(\Rightarrow \bot\).
\end{proof}

\begin{lemma} \label{lm2_3}
	\(A \in \operatorname{M}_n(\mathbb{F}),\ \mathbb{F}\) - алгебраически замкнуто. \(V \subseteq \mathbb{F}^n A\) - инвариантно. Тогда в \(V\) содержится собственный вектор матрицы \(A\).
\end{lemma}

\begin{proof}[Доказательство]
	Пусть \(\operatorname{dim}V = k > 0\) и \(e_1, \dots, e_k\) - базис \(V\). Дополним \(e_1, \dots, e_k\) до базиса всего \(\mathbb{F}^n : e_1, \dots, e_k, \dots, e_n\). Тогда:
	\[A = 
		\left(
			\begin{array}{c|c}
			A_{0_{k, k}} & B_{k, n - k}\\
			\hline
			0_{n - k, k} & C_{n - k, n - k}
			\end{array}
		\right) \text{, и } V = \left\{
			\left(
			\begin{array}{c}
				a_1\\
				\vdots\\
				a_k\\
				\vdots\\
				a_n
			\end{array}
			\right) \in \mathbb{F}^n
		\right\}
	\]
	\(\mathbb{F}\) - алгебраически замкнуто \(\Rightarrow \exists v_0 \in \mathbb{F}^k : A_ov_o = \lambda v_o\)
	\[v_o = \left(
		\begin{array}{c}
			b_1\\
			\vdots\\
			b_k
		\end{array}
	\right) \Rightarrow \exists v \in \mathbb{F}^n : v = \left(
		\begin{array}{c}
			b_1\\
			\vdots\\
			b_k\\
			0\\
			\vdots\\
			0
		\end{array}
	\right) \Rightarrow Av = \lambda v\]
\end{proof}

\begin{theorem} \label{th2_1}
	(Бёрнсайда) Пусть \(\mathbb{F}\) алгебраически замкнуто, \(\mathcal{A}\) --- неприводимая подалгебра \(\operatorname{M}_n(\mathbb{F}) \Rightarrow \mathcal{A} = \operatorname{M}_n(\mathbb{F})\)
\end{theorem}

\begin{proof}[Доказательство]
	~\newline 
	1) Сначала покажем, что \(\exists T \in \mathcal{A} : \operatorname{rk} T = 1\)
	\\
	Пусть \(T \neq 0\) --- имеет минимальный ранг. Предположим, что \(\operatorname{rk} T > 1 \Rightarrow \exists u, v \in \mathbb{F}^n : Tu \text{ и } Tv\) --- линейно независимы. Из \hyperref[lm2_1]{\textbf{Леммы 1}}:
	\[\exists A \in \mathcal{A} : A \cdot Tv = u \label{1} \tag{1}\]
	\[T[\mathbb{F}^n] \text{ - образ оператора } T \text{ --- } TA \text{ - инвариантен} \tag{2} \label{2}\]
	Из \hyperref[lm2_3]{\textbf{Леммы 3}} и (\ref{2}) \(\Rightarrow\) что \(T[\mathbb{F}^n]\) содержит собственный вектор \(TA\) с собственным значением \(\lambda\), тогда:
	\[\exists w \in T[\mathbb{F}^n]: TAw = \lambda w\]
	\[(TA - \lambda E)w = 0 \label{3} \tag{3}\]
	\[(TA -\lambda E)T = T(AT - \lambda E)\]
	следовательно образ \((TA - \lambda E)T\) лежит в \(T[\mathbb{F}^n]\) но не равен ему в силу (\ref{3}) \(\Rightarrow \) \[\operatorname{rk}((TA - \lambda E)T) < \operatorname{rk}T\]
	С другой стороны, по (\ref{1}) верно что:
	\[(TA - \lambda E)Tv = TATv - \lambda Tv = Tu - \lambda Tv \neq 0 \Rightarrow \bot \Rightarrow \operatorname{rk}T = 1\]
	2) Без ограничения общности в \(T\) первый столбец ненулевой \(\Rightarrow \exists C \in \mathcal{A}:\)
	\[\text{первый столбец \(CT\) равен } \left(
		\begin{array}{c}
			1\\
			0\\
			\vdots\\
			0
		\end{array}
	\right) \text{, но } \operatorname{rk}T = 1 \Rightarrow CT = \left(
		\begin{array}{cccc}
			1&a_{12}&\hdots&a_{1n}\\
			0&0&\hdots&0\\
			\vdots&\vdots&\ddots&\vdots\\
			0&0&\hdots&0
		\end{array}
	\right)\]
	Из \hyperref[lm2_2]{\textbf{Леммы 2}}:
	\[\exists D \in \mathcal{A}: CTD = \left(
		\begin{array}{cccc}
			1&0&\hdots&0\\
			\vdots&\vdots&\ddots&\vdots\\
			0&0&\hdots&0
		\end{array}
	\right) = E_{11}\]
	Аналогично, для любых \(i, j = \overline{1, n}: E_{ij} \in \mathcal{A} \Rightarrow \mathcal{A} = \operatorname{M}_n(\mathbb{F})\)
\end{proof}

\newpage

\section{Третья лекция}
\paragraph{Следствия из теоремы Бёрнсайда}
\begin{conseq}
	Пусть \(S = \{A_1, \dots, A_n\} \subseteq \operatorname{M}_n(\mathbb{F}),\ E \in \mathcal{L}(S),\ \mathbb{F}\) - алгебраически замкнуто. Тогда \(\mathcal{L}(S) = \operatorname{M}_n(\mathbb{F}) \Leftrightarrow S\) - неприводимо.
\end{conseq}

\begin{proof}[Доказательство]
	~\newline
	\[S \text{ - приводимо} \Leftrightarrow \exists \text{ базис, в котором } \forall A \in S:\
	A = \left(
		\begin{array}{c|c}
			A_0&B\\
			\hline
			0&C
		\end{array}
	\right) \Leftrightarrow \] 
	\[\Leftrightarrow \exists \text{ базис, в котором } \forall A \in \mathcal{L}(S): A = \left(
		\begin{array}{c|c}
			A_0&B\\
			\hline
			0&C
		\end{array}
	\right) \Rightarrow \mathcal{L}(S) \neq \operatorname{M}_n(\mathbb{F})\]
\end{proof}

\begin{conseq}
	Пусть \(\mathcal{A}\) --- подалгебра в \(\operatorname{M}_n(\mathbb{F}),\ E \in \mathcal{A},\ \mathbb{F}\) - алгебраически замкнуто, \(\mathcal{A} \neq \operatorname{M}_n(\mathbb{F})\). Тогда \(\operatorname{dim} \mathcal{A} \leqslant n^2 - n + 1\)
\end{conseq}

\begin{proof}[Доказательство]
	\(\mathcal{A} \neq \operatorname{M}_n(\mathbb{F}) \Rightarrow \exists \mathcal{A}\) - инвариантное подмонжество размерности \(k \Rightarrow\)
	\[\forall A \in \mathcal{A} : A = \left(
		\begin{array}{c|c}
			A_{0_{k, k}}&B_{k, n - k}\\
			\hline
			0_{n - k, k}&C_{n - k, n - k}
		\end{array}
	\right) \Rightarrow \forall k \in [1, n - 1] :\]
	\[\operatorname{dim} \mathcal{A} \leqslant n^2 - k(n - k) \leqslant \max\limits_{k \in [1, n - 1]} \left(n^2 - k(n - k) \right) \leqslant n^2 - n + 1\]
\end{proof}

\begin{note} \label{nt3_1}
	Пусть все матрицы в алегбре \(\mathcal{A}\) имеют вид:
	\[A = \left(
		\begin{array}{c|c|c|c}
			A_{11} & * & \hdots & * \\
			\hline
			0 & A_{22} & \ddots & \vdots \\
			\hline
			\vdots & \vdots & \ddots & * \\
			\hline
			0 & 0 & \hdots & A_{kk} \\
		\end{array}
	\right)\]
	Тогда:
	\[\forall i : \mathcal{A}_i = \{A|_{= i}\ |\ A \in \mathcal{A} \} \text{ - подалгебра в } \operatorname{M}_{n_i}(\mathbb{F}).\]
\end{note}

\begin{theorem} \label{th3_1}
	(о блочном строении) Пусть \(\mathcal{A}\) --- подалегбра \(\operatorname{M}_n(\mathbb{F}),\ E \in \mathcal{A},\ \mathbb{F}\) - алгебраически замкнуто. Тогда \(\exists\) базис в котором \(A \in \mathcal{A}\) имеет вид:
	\[A = \left(
	\begin{array}{c|c|c|c}
		A_{11} & * & \hdots & * \\
		\hline
		0 & A_{22} & \ddots & \vdots \\
		\hline
		\vdots & \vdots & \ddots & * \\
		\hline
		0 & 0 & \hdots & A_{kk} \\
	\end{array}
	\right)\]
	При этом:
	\[\forall i : \mathcal{A}_i = \{A|_{=i}\ |\ A \in \mathcal{A}\} = \operatorname{M}_{n_i}(\mathbb{F}).\]
\end{theorem}

\begin{proof}[Доказательство]
	Рассмотрим все цепочки вложенных инвариантных подпространств:
	\[\{0\} = V_0 \subset V_1 \subset V_2 \subset \dots \subset V_k = \mathbb{F}^n\]
	Выберем среди них цепочку максимальной длины. Построим базис в \(\mathbb{F}^n : \{e_1, \dots, e_n\}\).
	Рассмотрим базис в пространстве \(V_1 : \{e_1, \dots, e_{i_1}\}\), дополним его до базиса в \(V_2 : \{e_1,\dots, e_{i_1}, e_{i_1 + 1}, \dots, e_{i_2}\}\).
	Далее продолжим эту процедуру, пока не получим требуемый базис в \(\mathbb{F}^n\). В построенном базисе \(\{e_1, \dots, e_n\}:\)
	\[\forall A \in \mathcal{A}: A = \left(
	\begin{array}{c|c|c|c}
		A_{11} & * & \hdots & * \\
		\hline
		0 & A_{22} & \ddots & \vdots \\
		\hline
		\vdots & \vdots & \ddots & * \\
		\hline
		0 & 0 & \hdots & A_{kk} \\
	\end{array}
	\right)\]
	Пусть \(\exists m : \mathcal{A}_m = \operatorname{M}_{n_m}(\mathbb{F}) \Rightarrow \) по \hyperref[th2_1]{\textbf{Теореме Бёрнсайда}} и \hyperref[nt3_1]{\textbf{Замечанию 3}} \(\exists\) базис \(\mathbb{F}^n\), в котором \(\forall A \in \mathcal{A}_m\) имеет вид:
	\[\left(
		\begin{array}{c|c}
			A_{0_{i_m, i_m}} & B_{i_m, n - i_m}\\
			\hline
			0_{n - i_m, i_m} & D_{n - i_m, n - i_m}
		\end{array}
	\right)\]
	Пусть \(C\) - матрица перехода к этому базису, тогда рассмотрим следущий базис в \(\mathbb{F}^n :\)
	\[(g_1, \dots, g_n) = (e_1, \dots, e_n) \cdot H = (e_1, \dots, e_n) \left(
		\begin{array}{c|c|c}
			E & 0 & 0 \\
			\hline
			0 & C & 0 \\
			\hline
			0 & 0 & E
		\end{array}
	\right)\]
	Тогда, матрица \(A \in \mathcal{A}\) под действием \(H\) перейдет в базис \(\{g_1, \dots, g_n\}:\)
	\[H^{-1}AH = \left(
	\begin{array}{c|c|c|c}
		A_{11} & * & \hdots & * \\
		\hline
		0 & A_{22} & \ddots & \vdots \\
		\hline
		\vdots & \vdots & \ddots & * \\
		\hline
		0 & 0 & \hdots & A_{kk} \\
	\end{array}
	\right) = \left(
		\begin{array}{c|c|c|c|c|c}
			A_{11} & * & \hdots & \hdots & \hdots & *\\
			\hline
			0 & A_{22} & \ddots &&& \vdots \\
			\hline
			\vdots & 0 & \ddots & \ddots && \vdots \\
			\hline
			\vdots & \vdots && C^{-1} A_{mm} C & \ddots & \vdots \\
			\hline
			\vdots & \vdots &&& \ddots & * \\
			\hline
			0 & 0 & \hdots & \hdots & \hdots & A_{kk}
		\end{array}
	\right) = \]
	\[= \left(
	\begin{array}{c|c|c|c|c|c}
		A_{11} & * & \hdots & \hdots & \hdots & *\\
		\hline
		0 & A_{22} & \ddots &&& \vdots \\
		\hline
		\vdots & 0 & \ddots & \ddots && \vdots \\
		\hline
		\vdots & \vdots && \begin{array}{c|c}
			A_0 & B \\
			\hline
			0 & D
		\end{array} & \ddots & \vdots \\
		\hline
		\vdots & \vdots &&& \ddots & * \\
		\hline
		0 & 0 & \hdots & \hdots & \hdots & A_{kk}
	\end{array}
	\right)\]
	\[V_m^{\prime} = \{e_1, \dots, e_{i_{m-1}}, g_{i_{m - 1} + 1}, \dots, g_{i_m}, \dots, e_n\} \Rightarrow\]
	\[\Rightarrow \exists \text{ цепочка } \{0\} = V_0 \subset V_1 \subset V_2 \subset \dots \subset V_{m - 1} \subset V_m \subset V^{\prime}_m \subset V_{m + 1} \subset \dots \subset V_k = \mathbb{F}^n\]
	но эта цепочка длиннее, так как \(V_m \neq V^{\prime}_m \Rightarrow \bot\)
\end{proof}

\begin{exec}
	Пусть в терминах \hyperref[th3_1]{\textbf{Теоремы о блочном строении}} подалгебра \(\mathcal{A} \subset \operatorname{M}_6(\mathbb{F})\) имеет вид:
	\[\mathcal{A} = \left(
		\begin{array}{c|c}
			A_{1_{3,3}} & 0 \\
			\hline
			0 & A_{2_{3, 3}}
		\end{array}
	\right).\]
	Следует ли из этого, что \(\mathcal{A} = \operatorname{M}_3(\mathbb{F}) \oplus \operatorname{M}_3(\mathbb{F})\)?
\end{exec}

\begin{proof}[Решение]
	Нет, так как \hyperref[th3_1]{\textbf{Теорема о блочном строении}} ничего не говорит о линейной независимости блоков, и вообще говоря, это не правда.
\end{proof}

\begin{exec}
	Пусть в терминах \hyperref[th3_1]{\textbf{Теоремы о блочном строении}} алгебра
	\[\mathcal{A} = \left( 
		\begin{array}{c|c}
			A_1 & \begin{array}{cc}
				0&0\\
				0&0
			\end{array} \\
			\hline
			\begin{array}{cc}
				0&0\\
				0&0
			\end{array} & A_2
		\end{array}
	\right) \text{ и } \exists A = \left(
		\begin{array}{c|c}
			A_1 & 0 \\
			\hline
			0 & 0
		\end{array}
	\right) \in \mathcal{A} : A_1 \neq E\]
	Доказать, что \(\mathcal{A} = \operatorname{M}_2(\mathbb{F}) \oplus \operatorname{M}_2(\mathbb{F})\).
\end{exec}

\begin{proof}
	По \hyperref[th3_1]{\textbf{Теореме о блочном строении}}: \(\mathcal{A}_1 = \operatorname{M}_2(\mathbb{F}) = \mathcal{A}_2 \Rightarrow\)
	\[\mathcal{B} = \left\{
		\left(
			\begin{array}{c|c}
				\begin{array}{cc}
					0 & 0 \\
					1 & 0
				\end{array} & 0 \\
				\hline
				0 & C_1
			\end{array}
		\right);
		\left(
		\begin{array}{c|c}
			\begin{array}{cc}
				1 & 0 \\
				0 & 0
			\end{array} & 0 \\
			\hline
			0 & C_1
		\end{array}
		\right);
		\left(
		\begin{array}{c|c}
			\begin{array}{cc}
				0 & 1 \\
				0 & 0
			\end{array} & 0 \\
			\hline
			0 & C_1
		\end{array}
		\right);
		\left(
		\begin{array}{c|c}
			\begin{array}{cc}
				0 & 0 \\
				0 & 1
			\end{array} & 0 \\
			\hline
			0 & C_1
		\end{array}
		\right);
	\right.\]
	\[\left.
		\left(
			\begin{array}{c|c}
				C_5 & 0 \\
				\hline
				0 & \begin{array}{cc}
					0 & 1 \\
					0 & 0
				\end{array}
			\end{array}
		\right);
		\left(
		\begin{array}{c|c}
			C_5 & 0 \\
			\hline
			0 & \begin{array}{cc}
				0 & 0 \\
				0 & 1
			\end{array}
		\end{array}
		\right);
		\left(
		\begin{array}{c|c}
			C_5 & 0 \\
			\hline
			0 & \begin{array}{cc}
				0 & 0 \\
				1 & 0
			\end{array}
		\end{array}
		\right);
		\left(
		\begin{array}{c|c}
			C_5 & 0 \\
			\hline
			0 & \begin{array}{cc}
				1 & 0 \\
				0 & 0
			\end{array}
		\end{array}
		\right)
	 \right\} \subseteq \mathcal{A}\]
	 Если показать, что \(E_1 = \left(
	 \begin{array}{c|c}
	 	\begin{array}{cc}
	 		1 & 0 \\
	 		0 & 1
	 	\end{array} & 0 \\
	 	\hline
	 	0 & C_1
	 \end{array}
	 \right) \in \mathcal{A}\), то базис в \(\operatorname{M}_2(\mathbb{F}) \oplus \operatorname{M}_2(\mathbb{F})\) выглядит как \(\{E_1B_1, \dots, E_4B_4, B_5, \dots, B_8\}\)
	 
	 \begin{enumerate}
	 	\item \(\operatorname{rk} A_1 = 2\):
	 	\[\mathcal{A}_1 = \operatorname{M}_2(\mathbb{F}) \Rightarrow X = \left(
	 		\begin{array}{c|c}
	 			A_1^{-1} & 0 \\
	 			\hline
	 			0 & D
	 		\end{array}
	 	\right) \in \mathcal{A} \Rightarrow XA = E_1 \in \mathcal{A}_1\]
	 	\item \(\operatorname{rk} A_1 = 1\): аналогично доказательству \hyperref[th2_1]{\textbf{Теоремы Бёрнсайда}}, покажем что:
	 	\[\left(
	 		\begin{array}{c|c}
	 			\begin{array}{cc}
	 				1&0\\
	 				0&0
	 			\end{array} & 0\\
	 			\hline
	 			0 & 0
	 		\end{array}
	 	\right) \text{ и } \left(
	 	\begin{array}{c|c}
	 		\begin{array}{cc}
	 			0&0\\
	 			0&1
	 		\end{array} & 0\\
	 		\hline
	 		0 & 0
	 	\end{array}
	 	\right) \in \mathcal{A} \Rightarrow E_1 \in \mathcal{A}\]
	 	
	 \end{enumerate}
\end{proof}

\newpage

\section{Четвертая лекция}
\paragraph{Триангулизуемые множества матриц}

\begin{definition}
	\(\operatorname{T}_n(\mathbb{F})\) - множество верхнетреугольных матриц порядка \(n\)
\end{definition}

\begin{definition}
	Множество матриц \(\mathcal{A} \subseteq \operatorname{M}_n(\mathbb{F})\) триангулизуемо, если 
	\[\exists C \in \operatorname{GL}_n(\mathbb{F}) : \forall A \in \mathcal{A} : C^{-1}AC \in \operatorname{T}_n(\mathbb{F})\]
\end{definition}

\begin{note}
	Свойства инвариантные базису:
	\begin{itemize}
		\item \(\operatorname{rk}A, \operatorname{tr}A, \operatorname{det}A\)
		\item \(\chi_A(t), \mu_A(t)\)
		\item Спектр матрицы
		\item Коммутативность, нильпотентность
	\end{itemize}
\end{note}

\paragraph{Критерий триануглизуемости}

\begin{lemma}\label{lm4_1}
	Пусть \(A, B \in \operatorname{M}_n(\mathbb{F}), \lambda, \mu \in \mathbb{F}\), тогда:
	\[[A, B] = [A + \lambda E, B + \mu E]\]
\end{lemma}

\begin{lemma}\label{lm4_2}
	Пусть \(A \in \operatorname{T}_n(\mathbb{F})\). Тогда \(A\) --- нильпотентна \(\Leftrightarrow \forall i \in [1, n] : a_{ii} = 0\)
\end{lemma}

\begin{theorem}\label{th4_1}
	Пусть \(\mathbb{F}\) --- алегбраически замкнуто. Алгебра \(\mathcal{A} \subseteq \operatorname{M}_n(\mathbb{F})\) триангулизуема, тогда и только тогда, когда:
	\[\forall A, B \in \mathcal{A} : [A, B] \text{ --- нильпотентная}\]
\end{theorem}

\begin{proof}[Доказательство]
	~\\
	\((\Rightarrow)\) Выберем базис так, что: \(\forall A, B \in \mathcal{A}\) на диагонали матрицы \([A, B]\) стоят \(a_{ii}b_{ii} - b_{ii}a_{ii} = 0\), но \([A, B] \in \operatorname{T}_n(\mathbb{F}) \Rightarrow\), по \hyperref[lm4_2]{\textbf{Лемме 5}} \([A, B]\) --- нильпотентный\\
	\((\Leftarrow)\) Рассмотрим алгебру \(\mathcal{B} = \mathcal{L}(\mathcal{A} \cup \{E\})\). По \hyperref[th3_1]{\textbf{Теореме о блочном строении}}: существует базис в котором \(\mathcal{B}\) имеет блочно-верхнетреугольный вид:
	\[\left(
		\begin{array}{ccc}
			\mathcal{B}_{11} & \hdots & * \\
			\vdots & \ddots & \vdots \\ 
			0 & \hdots & \mathcal{B}_{kk}
		\end{array}
	\right),\ \forall i : \mathcal{B}|_i = \operatorname{M}_{n_i}(\mathbb{F})\]
	Пусть \(\exists i : n_i \geqslant 2.\) Тогда \(\exists A, B \in [A, B]\left|_i\right.\) не являющийся нильпотентным \(\Rightarrow [A, B]\) не нильпотентен.
	\[\forall A, B \in \mathcal{B}: \exists C, D \in \mathcal{A} \text{ и } \lambda, \mu \in \mathbb{F} : \]
	\[A = C + \lambda E,\ B = D + \mu E \Rightarrow \]
	\(\Rightarrow\) По \hyperref[lm4_1]{\textbf{Лемме 4}}: \([C, D] = [A, B]\) не нильпотентный \(\Rightarrow \bot \Rightarrow \forall i : n_i = 1 \Rightarrow \\ \Rightarrow \mathcal{B}\) --- триангулизуема \(\Rightarrow \mathcal{A}\) --- триангулизуема.
\end{proof}

\paragraph{Коммутативные и нильпотентные алгебры}

\begin{conseq}
	Пусть \(\mathbb{F}\) --- алгебраически замкнуто, \(S \subseteq \operatorname{M}_n(\mathbb{F}), \forall A, B \in S : AB = BA\). Тогда \(S\) --- неприводимая.
\end{conseq}

\begin{proof}[Доказательство]
	\(\mathcal{L}(S)\) - коммутативная алгебра \(\Rightarrow \forall A, B \in \mathcal{L}(S) : [A, B] = 0 \Rightarrow \) по \hyperref[th4_1]{\textbf{Теореме 3}} \(\mathcal{L}(S)\) - триангулизуема.
\end{proof}

\begin{conseq}
	Пусть \(\mathbb{F}\) - алгебраически замкнуто. Тогда \(\mathcal{A} \subseteq \operatorname{M}_n(\mathbb{F})\) нильпотентная алгебра(те, \(\forall A \in \mathcal{A} : \exists n : A^n = 0\)), тогда \(\mathcal{A}\) --- триангулизуема.
\end{conseq}

\begin{proof}[Доказательство]
	\(\forall A, B \in \mathcal{A} \Rightarrow [A, B]\) нильпотентен \(\Rightarrow\) по \hyperref[th4_1]{\textbf{Теореме 3}}: \(\mathcal{A}\) --- триангулизуема.
\end{proof}

\begin{conseq}
	\(\mathbb{F}\) --- алгебраически замкнутое поле. Тогда \(\mathcal{A} \subseteq \operatorname{M}_n(\mathbb{F})\) нильпотентная алгебра \(\Leftrightarrow \exists\) базис в котром \(\forall A \in \mathcal{A} : A\) строго верхнетреугольная.
\end{conseq}

\begin{definition}
	Матрицы \(A, B\) одновременно триангулизуемы, если \(\{A, B\}\) триангулизуемо.
\end{definition}

\begin{theorem}
	(Маккоя, изначальный вариант) \(A, B\) одновременно триангулизуемы \(\Leftrightarrow\)
	\[\Leftrightarrow \forall C \in \mathcal{L}(\{A, B, E\}) : C[A, B] \text{ --- нильпотентен.}\]
\end{theorem}

\begin{proof}[Доказательство]
	~\\
	\((\Rightarrow)\) Аналогично, доказательству \hyperref[th4_1]{\textbf{Теоремы 3}}.\\
	\((\Leftarrow)\) Рассмотрим \(\mathcal{A} = \mathcal{L}(\{A, B, E\})\). По \hyperref[th3_1]{\textbf{Теореме о блочном строении}}, \(\mathcal{A}\) имеет вид:
	\[\left(
		\begin{array}{ccc}
			\mathcal{A}_{11} & \hdots & * \\
			\vdots & \ddots & \vdots \\
			0 & \hdots & \mathcal{A}_{kk}
		\end{array}
	\right),\ \forall i : \mathcal{A}_i = \operatorname{M}_{n_i}(\mathbb{F})\]
	Пусть, \(\exists i : n_i \geqslant 2: AB|_i \neq BA|_i\) (иначе \(\operatorname{M}_{n_i}(\mathbb{F}) = \mathcal{A}_i = \mathcal{L}(\{A, B, E\})|_i\) --- коммутативная алгебра), тогда \[\exists x \in \mathbb{F}^{n_i} : [A, B] \left|_i \right. \cdot x \neq 0\]
	\[\mathcal{L}(\{A, B, E\}) \left|_i \right. = \operatorname{M}_{n_i}(\mathbb{F}) \Rightarrow \exists C \in \mathcal{L}(\{A, B, E\}):\]
	\[C \left|_i \right. \cdot [A, B] \left|_i \right. \cdot x = x \Rightarrow \forall n : (C \cdot [A, B])^n \cdot x = x \neq 0\]
	Следовательно, \(C \cdot [A, B]\) не нильпотентен \(\Rightarrow \bot \Rightarrow \mathcal{A}\) --- триангулизуема \(\Rightarrow A, B\) --- одновременно триангулизуемы.
\end{proof}

\newpage
\section{Используемые обозначения и соглашения}

\begin{itemize}
	\item В курсе в основном рассматриваются ассоциативные алгебры
	\item \([A, B]\) - коммутатор матриц \(A, B\). \([A, B] = AB - BA\)
	\item \(B \left|_{x} \right.\) - ограничение на блочную матрицу или на блочную алгебру. Имеется ввиду блок из матрицы(алгебры) с индексом \(x\)
\end{itemize}

\end{document}
